\chapter*{Zadání práce a úvod}
\addcontentsline{toc}{chapter}{Zadání práce a úvod}
Předmětem práce je program, jenž v~jednoduchém uživatelském prostředí umožní uživateli zobrazit grafy různých matematických funkcí s~jednou neznámou.
Program dokáže zobrazit hned několik grafů na jednou a~vykreslené grafy může uložit do souboru jako obrázek.
Uživatel má také možnost graf lobovolně přiblížit, nebo oddálit.
Lze také nastavit libovolnou barvu grafu, pokud jí uživatel nezvolí použije se jedna z~10 předefinovaných pastelových barev.
Společně s~grafem funkce se také vykresluje měřítko obou os, to je stejné pro obě osy.
Měřítko se mění automaticky společně s~přiblížením grafu.

Program je napsaný v~jazyce Java a~na grafické zobrazení používá knihovnu JavaFX, při exportování a~importování do/z souboru je použit JSON, s~ním se operuje prostřednictvím knihovny json.
Bonus jsem při vypracovávání práce stanovené neměl, funkce jsem přidával v~pořadí jak mě napadly.

